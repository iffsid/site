\documentclass[10pt]{article}
\usepackage{charter}
\usepackage{textcomp}
\usepackage{stmaryrd}
\usepackage{fullpage}
\usepackage[usenames,dvipsnames,svgnames]{xcolor}
\usepackage{hyperref}
\usepackage[flushleft,neverdecrease,neveradjust]{paralist}
\pagestyle{empty}
\raggedbottom
\raggedright

\textheight=9.0in
\setlength{\tabcolsep}{0in}
\setlength{\oddsidemargin}{-0.5cm}
\setlength{\evensidemargin}{-0.5cm}
\setlength{\textwidth}{7.0in}
\usepackage[T1]{fontenc}

\renewcommand{\labelitemi}{}
\renewcommand{\labelitemii}{}

\hypersetup{
  colorlinks=true,
  urlcolor=Sepia,
  pdfborder= 0 0 0,
  bookmarks=false,
  pdftitle={Siddharth Narayanaswamy - Research Statement},
  pdfauthor={Siddharth Narayanaswamy},
  pdfsubject={Research Statement}}

\newenvironment{researchBlock}[1]{%
  \vspace*{0.5ex}
  {\large \textbf{#1}}
  \begin{enumerate}\item[]}
  {\end{enumerate}}

\newcommand{\refr}[1]{{\color{RoyalBlue} #1}}
\newcommand{\LincolnLogs}{\textsc{Lincoln Logs}}
\newcommand{\LincolnLog}{\textsc{Lincoln Log}}

\defaultleftmargin{10pt}{}{}{}

\begin{document}

\begin{tabular*}{6.86in}{l@{\extracolsep{\fill}}r}
  \textbf{\huge{Research Statement}} & \today
\end{tabular*}
\vspace{0.1in}

\begin{tabular*}{6.86in}{l@{\extracolsep{\fill}}r}
  \textbf{\large{Siddharth Narayanaswamy}} & \url{http://www.iffsid.com}
\end{tabular*}
\vspace{0.4in}

\begin{researchBlock} {Reasoning about part-based 3D structures}
  %
  Many physical entities in the world exhibit a compositional structure, in
  that they are composed of different kinds of parts, arranged in certain
  fashions.
  %
  As humans, we are able to perceive the underlying structure of such entities,
  interact with them, and describe them where necessary.
  %
  In order to do so, we collate information from and across vision, language,
  and action, in the context of what we know about the physics of the world.
  %
  The fact that we are able to achieve this, is however, no indication of the
  ease of the problem; quite the opposite, in fact.
  %
  The general dearth of distinctive visual features and an overabundance of
  occlusion combine to exacerbate the difficulty of the process.

  We develop an integrated and cross-modal approach which can percieve,
  manipulate, and describe part-based 3D structures.
  %
  It combines knowledge about the different kinds of parts used to construct a
  structure and knowledge about how such parts may combine physically to form
  stable configurations, to reason about the complete part-by-part composition
  of such a structure.
  %
  In doing so, it incorporates reasoning about occlusion, in that reasoning
  about the part-by-part composition of 3D structures necessitates reasoning
  about the parts of it that cannot be seen due to occlusion.
  %


  All of this is facilitated by representing the knowledge about the
  parts, how they combine, and some basic world knowledge in a large graphical
  model.
  %
  The compositional nature of



  This work involves reasoning about the physical structure of composable
  entities, involving integration within vision, and across vision and language
  through the medium of robotics.
  %
  The work is instantiated in the domain of \LincolnLogs, which allows for a
  combinatorially large number of assemblies from a relatively tiny inventory of
  1-, 2-, and 3-notch logs.
  %
  We estimate the composition of a given assembly of \LincolnLogs\ by reasoning
  about the unreliable low-level visual features in the context of high-level
  physical constraints of assembly (E.g., logs combine orthogonally and only at
  notches).
  %
  The framework allows for reasoning about occlusion, reasoning about an assembly
  from multiple views and partial disassembly, and reasoning about an assembly in
  the context of natural-language descriptions of the structure.
  %
  The linguistic framework derives the lexicon to describe a \LincolnLog\
  assembly from topology, Such integration in our framework is made easy by the
  fact that, in practice, they only involve expanding and contracting the overall
  constraint-satisfaction problem.

  Publications: \refr{J2}, \refr{C6}, and \refr{C3}

\end{researchBlock}

\begin{researchBlock} {Learning rules of games through perception}

  Children learn to play games by watching others play. While both formal board
  games such as Chess and Checkers, and less formal play such as Hopscotch and
  Tag, all have well defined rules that children ultimately come to know, they
  are rarely told those rules explicitly. This work features an integrated
  vision and robotic system that plays, and learns to play, simple
  physically-instantiated board games that are variants of TicTacToe and
  Hexapawn. The rules of the games are learned solely by observing physical
  play can subsequently be used to drive further physical play.

  Inspired by how children appear to learn rules about the world from
  demonstration and interaction, we model, in the domain of board games, an
  analogous system that learns the rules of board games from visual
  observation.
  %
  Here, two robotic agents, the protagonist and the antagonist play a physically
  instantiated board game.
  %
  A third agent, the wannabe, watches the gameplay and attempts to infer the
  rules of the game, given some minor background knowledge (what directions mean,
  etc.) about the world.
  %
  The physical instantiation forces the inference process, driven by Inductive
  Logic Programming (ILP), to happen from real-world input.
  %
  Using this frame- work, we learn the complete rules (initial board, legal move
  generator, and outcome predicate) of six games -- Tic-Tac-Toe, Hexapawn and 4
  variants thereof.
  %
  Furthermore, We introduce a natural-language component into the system, by
  giving the protagonist and the antagonist the rules of the game being played in
  English, and enabling translation from the internal representations to natural
  language.

  Publications: \refr{C2}
\end{researchBlock}

\begin{researchBlock} {Compositional representation of events in the human brain}

  In keeping with the overarching theme, we further explore compositionality in
  a fist-of-its-kind study where people in an fMRI machine are shown videos
  depicting activities that correspond to unique sentential descriptions. By
  classifying brain-activity with different subsets of the sentential
  structure, first independently and then jointly, we show that brain-activity
  patterns reflect compositionality in sentence structure as the composition of
  independent classifications matches those obtained jointly.

  We investigate the compositionality of argument structure, i.e., how
  participants fill roles in events, of sentences in the human brain, in a
  first-of-its-kind study where people in an fMRI machine are shown videos
  depicting activities that correspond to unique sentential descriptions.
  %
  We recover the compositional-semantic components independently from
  counterbalanced experiments on stimuli that vary multiple variables by
  collapsing along different variables, showing that brain-activity patterns
  reflect compositionality in sentence structure as the composition of
  independent experiments matches those obtained jointly.
  %
  Additionally, we also produce the first result on the classification of verbs
  in the human brain, performing with 80\% accuracy on a 1-out-of-6
  classification task involving the picked up, put down, carry, hold, walk, and
  dig events.
  %
  Furthermore, the robustness of these results were tested and verified by both
  testing across subjects and across sites (to account for variance in
  equipment).

  Publications: \refr{T2}
\end{researchBlock}

\begin{researchBlock} {Activity Reconition}

  Here, we show how the compositional structure of events, in concert with the
  compositional structure of language, can interplay with the underlying
  focusing mechanisms in video action recognition. Such an integrated framework
  allows for a multitude of tasks simply by leveraging its features in
  different ways. This allows, on videos depicting multiple activities, to
  perform taks such as sentential-query based video retrieval, sentential video
  description, and a natural-language guided focus-of-attention mechanism.

  We introduce a framework that integrates natural-language descriptions of
  activities with a detection-based tracking mechanism for activity recognition
  in a provably optimal fashion.
  %
  We intend for the sentences, describing interaction between participant
  objects, to provide high-level top-down guidance to the inherent tracking
  process in the activity recognition framework.
  %
  Given a video and a sentence as its input, it produces as output a score and a
  set of tracks that best represent the provided sentential description.
  %
  By leveraging this framework in different ways, we are able to demonstrate its
  efficacy and elegance in performing three disparate tasks: (a)
  sentential-description based focus of attention, (b) sentential-description
  generation for video, and (c) sentential-query based video retrieval.


  We modify and extend the Sentence Tracker’s ability to perform query-based
  video retrieval on a very large corpus of videos using machine-learning
  methods in order to learns models for the query lexicon.
  %
  We demonstrate how the roles that objects/participants play in the described
  activities affects the retrieval mechanism, and how the lack of such a
  mechanism is detrimental to the retrieval task.
  %
  Also, our tests are carried out with the help of off-the-shelf object
  detectors.
  %
  Moreover, we do so restricting our method to only use off-the-shelf pre-trained
  object detectors from the current state-of-the-art.
  %
  We do so to draw focus to the benefits of our underlying framework, avoiding
  the confounds of detectors trained in-house.
  %
  The scalability and efficacy of our framework is demonstrated by deploying its
  capabilities across a corpus of ten full-length Hollywood videos, amounting to
  approximately 25 hours of video, and comparing against existing
  state-of-the-art methods to do video retrieval.
  %
  The incorporation of query semantics provides us with a distinct advantage in
  such tasks.


  The central contribution of this body of work is the implementation and
  deployment of a real-time multi-class activity and description system.
  %
  We make use of our viterbi-tracker framework for activity recognition, running
  a multitude of such trackers in parallel on objects, using both our own
  in-house object detector and the Felzenszwalb object detector optimized for the
  GPU.
  %
  The in-house object detector is particularly effective in identifying rigid
  objects under arbitrary spatial transformations, with only a handful of
  training examples; a shortfall of existing state-of-the-art object detectors.
  %
  The system is capable of identifying actions such as picked up, put down,
  raised, lowered, give, carry, walked, hold, replaced, and exchanged.
  %
  On identifying the action, the system produces natural-language output, both
  orthographic and aural, describing any subset of the action, its constituents,
  the roles played by such, spatial relations, and the temporal profile of the
  action in sentential form.

  Publications: \refr{J3}, \refr{J1}, \refr{C7}, \refr{C5}, \refr{C4},
  \refr{T4}, and \refr{T3}
\end{researchBlock}

\end{document}
