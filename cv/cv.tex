\documentclass[10pt]{article}
\usepackage{textcomp}
\usepackage{stmaryrd}
\usepackage{fullpage}
\usepackage{hyperref}
\usepackage{url}
\pagestyle{empty}
\raggedbottom
\raggedright

\textheight=9.5in
\setlength{\tabcolsep}{0in}
\setlength{\oddsidemargin}{-0.8cm}
\setlength{\evensidemargin}{-0.8cm}
\setlength{\textwidth}{7.0in}
\setlength{\topmargin}{-1.0cm}
\usepackage[T1]{fontenc}
\usepackage{cmbright}
%% \usepackage[sc]{mathpazo}
\linespread{1}

\renewcommand{\labelitemi}{}
\renewcommand{\labelitemii}{}

\hypersetup{
  colorlinks=false,
  urlcolor=black,
  pdfborder= 0 0 0,
  bookmarks=false,
  pdftitle={Siddharth Narayanaswamy - Resume},
  pdfauthor={Siddharth Narayanaswamy},
  pdfsubject={Resume},
  pdfkeywords={Siddharth Narayanaswamy, resume, cv, mathematics, cs,
    computer science, phd, computer vision, vision, perception, robots,
    robotics, image processing, speech processing, AI, Artificial Intelligence,
    machine learning, open source, open-source, functional programming,
    constraint, constraint satisfaction, automatic differentiation, stochastic
    programming, C++, Scheme, Lisp, C, Prolog, MATLAB, Java, Python, purdue}}

\newenvironment{position}[4]
{\filbreak
\item
  \begin{tabular*}{6.5in}{l@{\extracolsep{\fill}}r}
    \textbf{#1} & \textit{#2} \\ #3 & \footnotesize{#4} \\
  \end{tabular*}
  \begin{itemize} \setlength{\parskip}{-1pt}}
  { \end{itemize} }

\newenvironment{publication}[6]
{ \item
  \begin{tabular*}{6.5in}{p{5.2in}@{\extracolsep{\fill}}r}
    \textbf{#1} & \textit{#2}\\ #3 & \footnotesize{#4}\\ \textit{#5}\\ #6
  \end{tabular*}
} {}

\newcommand{\reference}[7]{%
  \begin{tabular}{@{}l@{}}%
    \textbf{#1}\\%
    #2\\%
    #3\\%
    #4\\%
    #5\\%
    #6\\%
    \href{mailto:#7}{#7}%
  \end{tabular}%
}

%% -- For Hevea translation (latex -> html)
%% -- It has difficulty expanding macros unless completely contained
%% -- also need to comment out \url
%% -- hevea resume.tex -O -fix -o resume.html
% \newenvironment{position}[4]
% {\filbreak
% \begin{itemize}
% \item
%   \begin{tabular*}{6.5in}{l@{\extracolsep{\fill}}r}
%     \textbf{#1} & \textit{#2} \\ #3 & \footnotesize{#4} \\
%   \end{tabular*}
%   \begin{itemize} \setlength{\parskip}{-1pt}}
%     { \end{itemize} \end{itemize} }

% \newenvironment{publication}[6]
% { \begin{itemize}
% \item
%   \begin{tabular*}{6.5in}{p{5.2in}@{\extracolsep{\fill}}r}
%     \textbf{#1} & \textit{#2}\\ #3 & \footnotesize{#4}\\ \textit{#5}\\ #6
%   \end{tabular*}
% } {\end{itemize}}

\newenvironment{region}[1] {{\large \textbf{#1}} \begin{itemize}} {\end{itemize}}

\begin{document}

\begin{tabular*}{9in}{@{\extracolsep{\fill}}lcr}
  \textbf{\huge{Siddharth Narayanaswamy}} & \textrm{\today}
\end{tabular*}

\begin{tabular*}{6.85in}{@{\extracolsep{\fill}}lcr}
  \textrm{(323) 743-3239}& \textrm{siddharth@iffsid.com} & \url{http://iffsid.com}\\
\end{tabular*} \\
\vspace{+0.15in}

\begin{region}{Skills}
\item
  extensive experience with algorithms, computer-vision, signals~and~systems, robotics,\\
  machine learning, statistics, visual perception, cognitive science, linguistice,\\
  knowledge representation, compiler design, haptics, automatic-differentiation,\\
  stochastic and non-deterministic programming languages, functional programming,\\
  logic and constraint programming, parallel programming, contributions to open-source projects
\item \textbf{Languages} \\
  Scheme, Haskell, Lisp, C/C++, MATLAB, Python, Prolog, Java, VHDL, \\
  Verilog, DSP(AD Blackfin) assembly, microcontroller assembly, x86 assembly \\
  \vspace*{1ex}
  native English, Tamil, and Hindi
\end{region}

\begin{region} {Experience}
  \begin{position} {PhD student, Artificial Intelligence}
    {2008 -- present}
    {Jeffrey Mark Siskind}
    {Purdue University}
  \item designed and built custom robots for general manipulation tasks
  \item solved vision and robotic manipulation problems using AD-based optimization
  \item implemented closed-loop visual-servoing mechanism to drive motor control
  \item developed and implemented stochastic programs to use language \& vision,\\
    \  and reasoning about rules jointly, to solve for perception
  \item developed novel and robust tracking, segmentation, and action recognition\\
    \  methods as part of the DARPA Mind's Eye program
  \item nondeterministic programming for solving constraint-satisfaction problems
  \item stochastic modeling via probabilistic programming
  \item designed and evaluated human-subject experiments on event recognition\\
    \  using fMRI data collection and analysis tools
  \item TA for ECE473 \& ECE570, Artificial Intelligence
  \item \url{http://iffsid.com/research/}
  \end{position}
  \begin{position} {Undergraduate Research Project}
    {May 2007 -- Aug. 2008}
    {Guided by Professor Muniyandi Manivannan}
    {IIT-Madras}
  \item implemented a DIY Part-Task Laparoscopic Simulator
  \item optimized vision algorithms to run on low-cost uncalibrated components
  \item tested against industry-standard equipment to demonstrate reasonably\\
    \  low error margins
  \item worked on a haptic-vision developmental interface
  \item collaborated with practicing doctors to test feasibility
  \end{position}
  \begin{position} {Part-Time Instructor}
    {Oct. 2007 -- Aug. 2008}
    {The Princeton Review(Manya Education Pvt.Ltd.)}
    {Chennai, India}
  \item for the GRE and GMAT standardized exams
  \item tutored around 250 students
  \end{position}
  \begin{position} {Undergraduate Research Intern}
    {May 2007 -- Sept. 2007}
    {Doors and Gates Pvt.Ltd.}
    {Chennai, India}
  \item implemented a range of IR control mechanisms for controller operations
  \item tested and used implemented mechanisms successfully in robots during competitions
  \item designed hybrid control mechanism for non-line-of-sight applications\\
    \  that included switching between multiple modes of operation autonomously
  \end{position}
\end{region}

\newpage
\begin{region} {Education}
  \begin{position}{Phd Student, Artificial Intelligence}
    {2008 -- present}
    {}
    {Purdue University}
    \vspace{-0.2in}
  \item Artificial Intelligence, Computer Vision, Natural Language Processing
  \item Machine Learning, Robotics, Cognitive Neuroscience
  \end{position}
  \begin{position} {Bachelor of Engineering - Electronics and Communication}
    {2004 -- 2008}
    {}
    {Anna University, India}
    \vspace{-0.2in}
  \item Image Processing, Speech Processing, Computer Vision
  \item Communication Systems, Embedded Systems (Robotics)
  \end{position}
\end{region}

\begin{region} {Publications / Posters}
  \begin{publication} {Recognizing Human Activities from Partially Observed Videos}
    {Jun 2013}
    {Y. Cao, D. Barrett, A. Barbu, N. Siddharth, H. Yu, A. Michaux, Y. Lin,
      S. Dickinson, J. Siskind, S. Wang}
    {Poster}
    {IEEE Conference on Computer Vision and Pattern Recognition (CVPR)}
    {}
  \end{publication}
  \begin{publication} {Seeing Unseeability to See the Unseeable}
    {Oct 2012}
    {N. Siddharth, A. Barbu, and J. M. Siskind}
    {Paper/Journal}
    {Advances in Cognitive Systems(ACS)}
    {}
  \end{publication}
  \begin{publication} {Simultaneous Object Detection, Tracking, and Event Recognition}
    {Oct 2012}
    {A. Barbu, N. Siddharth, A. Michaux, and J. M. Siskind}
    {Paper/Journal}
    {Advances in Cognitive Systems(ACS)}
    {}
  \end{publication}
  \begin{publication} {Video In Sentences Out}
    {Aug 2012}
    {A. Barbu, A. Bridge, Z. Burchill, D. Coroian and S. Dickinson , S. Fidler,
      A. Michaux, S. Mussman and N. Siddharth , D. Salvi, L. Schmidt, J. Shangguan and
      J. M. Siskind, J. Waggoner, S. Wang, J. Wei and Y. Yin and Z. Zhang}
    {Paper/Conference}
    {Proceedings of the Twenty-Eighth Conf. on Uncertainity in Artificial Intelligence(UAI)}
    {}
  \end{publication}
  \begin{publication} {A Visual Language Model for Estimating Object
      Pose and Structure in a Generative Visual Domain}
    {May 2011}
    {N Siddharth, A. Barbu, and J. M. Siskind}
    {Paper/Conference}
    {Proceedings of 2011 IEEE International Conf. on Robotics and Automation(ICRA)}
    {}
  \end{publication}
  \begin{publication}{Learning Physically-Instantiated Game Play Through Visual Observation}
    {May 2010}
    {A. Barbu, N Siddharth, and J. M. Siskind}
    {Paper/Conference}
    {Proceedings of 2010 IEEE International Conf. on Robotics and Automation(ICRA)}
    {}
  \end{publication}
  \begin{publication} {Design of a Do-It-Yourself VR Based Laparoscopic Simulator}
    {Jan. 2009}
    {N.Siddharth, M.Manivannan, Suresh Devasahayam, and George Mathew}
    {Poster}
    {Medicine Meets Virtual Reality (MMVR17)}
    {}
  \end{publication}
\end{region}

% \newpage
% \begin{region}{References}
% \item
%   \begin{tabular*}{0.95\textwidth}{@{}l@{\hspace*{2ex}}l@{}}
%     \reference{Jeffrey Mark Siskind}
%     {School of Electrical \& Computer Engineering}
%     {Purdue University}
%     {465 Northwestern Avenue}
%     {West Lafayette, IN 47907, USA}
%     {765-496-3197}
%     {qobi@purdue.edu}&
%     \reference{Patrick Winston}
%     {MIT Computer Science \& Artificial Intelligence Laboratory}
%     {The Stata Center, Building 32}
%     {32 Vassar Street}
%     {Cambridge, MA 02139, USA}
%     {617-253-6754}
%     {pwh@csail.mit.edu}\\
%   \end{tabular*}
% \end{region}

\end{document}
\url{http://iffsid.com/icra2010}